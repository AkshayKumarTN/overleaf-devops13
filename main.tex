\documentclass{article}
\usepackage{hyperref}
\usepackage{fancyhdr}
\usepackage{lastpage}
\usepackage{graphicx}
\usepackage{listings}
\usepackage{xcolor}

% Version info
\newcommand{\gitversion}{Version: draft (manual)}

\title{DevOps Assignment 7 \\ Automated LaTeX Compilation with GitHub Actions \\ \small \gitversion}
\author{Akshay Kumar - Group 13}
\date{\today}

\pagestyle{fancy}
\fancyhead[L]{DevOps Group 13}
\fancyhead[R]{\gitversion}
\fancyfoot[C]{Page \thepage\ of \pageref{LastPage}}

% Code listing style
\lstset{
    basicstyle=\ttfamily\small,
    breaklines=true,
    frame=single,
    backgroundcolor=\color{gray!10}
}

\begin{document}

\maketitle

\tableofcontents
\newpage

\section{Introduction}
This document demonstrates automated LaTeX compilation using GitHub Actions. Every time changes are pushed to the repository, a new PDF version is automatically generated with proper version tracking.

\subsection{Project Goals}
\begin{itemize}
    \item Automatic compilation on git push
    \item Version tracking with changelist numbers
    \item PDF storage in repository and as artifacts
    \item Self-hosted runner implementation
    \item Docker containerization of build environment
\end{itemize}

\section{GitHub Actions Workflows}

\subsection{Cloud Runner}
The cloud runner workflow uses GitHub's hosted Ubuntu runners to compile the LaTeX document. This provides a clean, consistent build environment without requiring local infrastructure.

\subsection{Self-Hosted Runner}
The self-hosted runner runs on our own Overleaf server, providing:
\begin{itemize}
    \item Faster compilation (no cold start)
    \item Access to local resources
    \item Custom LaTeX packages
    \item Full control over build environment
\end{itemize}

\section{Implementation Details}

\subsection{Version Tracking}
Each PDF includes a version number in the format: \texttt{CL\#\#\#-hash}
\begin{itemize}
    \item CL\#\#\#: Changelist number (total commit count)
    \item hash: Short git commit hash (7 characters)
    \item Date: Commit date
\end{itemize}

\subsection{Automated Workflow}
\begin{enumerate}
    \item Developer pushes changes to GitHub
    \item GitHub Actions triggers automatically
    \item Version information is generated
    \item LaTeX document is compiled
    \item PDF is saved to repository
    \item ZIP archive is created
    \item Artifacts are uploaded
\end{enumerate}

\section{Repository Structure}
\begin{lstlisting}
overleaf-devops13/
|-- main.tex              (This document)
|-- version.tex           (Auto-generated version info)
|-- .github/
|   `-- workflows/
|       |-- compile-latex.yml
|       |-- compile-and-commit.yml
|       `-- compile-local-runner.yml
|-- compiled-pdfs/        (Cloud-compiled PDFs)
|-- local-compiled/       (Self-hosted compiled PDFs)
`-- README.md
\end{lstlisting}

\section{Testing and Verification}
This section will be updated with test results as the system is validated.

\section{Conclusion}
Automated LaTeX compilation with GitHub Actions provides a robust, reproducible document build system. The combination of cloud and self-hosted runners offers flexibility and reliability.

\section{References}
\begin{itemize}
    \item GitHub Actions Documentation: \url{https://docs.github.com/en/actions}
    \item LaTeX Project: \url{https://www.latex-project.org/}
    \item Domain: devops-13-overleaf.me
    \item Repository: \url{https://github.com/AkshayKumarTN/overleaf-devops13}
\end{itemize}

\end{document}

\subsection{Testing Automatic Workflow}
This change should trigger the workflow automatically.